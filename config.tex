% =============================================================================
% PREAMBLE CONFIGURATION
% =============================================================================

\usepackage[portuguese]{babel}
\usepackage[utf8]{inputenc} % UTF-8 encoding for special characters

% --- Document Layout & Margins ---
\usepackage[
  left=30mm,
  right=55mm,          % Wide right margin for sidenotes and bibliography
  top=30mm,
  bottom=30mm,
  marginparwidth=40mm, % Width of the sidenotes/margin area
  marginparsep=6mm     % Distance between main text and margin area
]{geometry}

% --- Language and Fonts ---
\usepackage[T1]{fontenc}    % Use modern font encoding for special characters

% --- Mathematics Support ---
\usepackage{amsmath}  % Advanced math typesetting (align, etc.)
\usepackage{amssymb}  % Extended mathematical symbols

% --- Visual Styling ---
\usepackage{xcolor}   % Color support
\usepackage{graphicx} % Support for including images
\usepackage{caption}  % Customizing figure/table captions
\usepackage{multicol} % Support for multi-column layouts
\usepackage{fancyhdr} % Custom headers and footers
\usepackage{tocloft}  % Detailed control over Table of Contents styling

% --- Margin & Sidenote Management ---
\usepackage{marginnote} % Basic marginal notes
\usepackage{marginfix}  % Prevents marginal notes from overlapping or overflowing pages
\setlength{\marginparpush}{8pt} % Minimum vertical space between margin elements
\usepackage{changepage} % Para expandir a largura
\usepackage{caption}    % Para captionof
\usepackage{tabularx}

% --- Bibliographic Management (Natbib + Usebib) ---
\usepackage[round, authoryear]{natbib} % Author-year citation style
\usepackage{bibentry}                  % Allows extracting full BibTeX entries into text
\usepackage{usebib}                    % Allows direct access to specific .bib fields (e.g., titles)
\bibinput{refs}                        % Loads the bibliography database (refs.bib)
\nobibliography* % Initializes bibentry system

% --- Utilities ---
\usepackage{lipsum}   % Dummy text generation for testing layout
\usepackage{etoolbox} % Programming tools for LaTeX macros
\usepackage{siunitx}

% --- Hyperlinks Configuration ---
\usepackage[
    colorlinks=true,
    linkcolor=blue,    % Color for internal links (ToC, sections)
    urlcolor=blue,     % Color for external URLs
    citecolor=blue,    % Color for standard citations
    anchorcolor=blue,
    bookmarksnumbered=false
]{hyperref}


% =============================================================================
% CUSTOM MACROS AND COMMANDS
% =============================================================================

% --- Table of Contents Styling ---
% Adds dots for Section entries and adjusts dot density
\renewcommand{\cftsecdotsep}{\cftdotsep} 
\renewcommand{\cftdotsep}{1}

\newcommand{\addtoc}{
    {
    \hypersetup{linkcolor=black} % Keep ToC links black for professional look
    \renewcommand{\contentsname}{Índice} % Força o nome aqui, se preferir
    \setcounter{tocdepth}{1}     % Show only Sections in the ToC
    \small
    \tableofcontents
    }
}

% --- Header and Footer (Fancyhdr com Extramarks) ---
\usepackage{extramarks} % Necessário para usar \firstleftmark

\pagestyle{fancy}
\fancyhf{} % Limpa os padrões
\renewcommand{\headrulewidth}{0pt} % Sem linha horizontal

% L: Imprime a primeira seção que aparecer na página. 
% Se não houver nova seção, mantém a atual.
\fancyhead[L]{\small\firstleftmark} 

\fancyhead[R]{\small\thepage} % Número da página à direita

% Ajuste para formatar o que vai para o mark (Número + Nome)
\renewcommand{\sectionmark}[1]{%
  \markboth{\thesection\enskip #1}{}%
}

% Estilo para páginas "plain" (início de capítulos ou onde houver \thispagestyle{plain})
\fancypagestyle{plain}{
  \fancyhf{} 
  \fancyfoot[C]{\small\thepage}
}

\usepackage[abspage]{zref}
\definecolor{darkgrey}{RGB}{60,60,60}

\makeatletter % Necessário para as macros que usam @

% 1. --- Custom Sidenote Commands ---
\newcounter{sidenote}
\newcommand{\sidenote}[1]{%
  \refstepcounter{sidenote}%
  \textsuperscript{\hyperref[sidenote:\thesidenote]{\textcolor{blue}{\thesidenote}}}%
  \marginpar{%
    \raggedright
    \scriptsize\textcolor{blue}{\thesidenote}.\,\textcolor{darkgrey}{#1}%
  }%
  \label{sidenote:\thesidenote}%
}

% 2. --- Professional Side Bibliography Macro (\sideref) ---
\newcounter{sideref}
\newcommand{\sideref}[1]{%
    \ifcsname myrefnum@#1\endcsname
        \edef\currentrefnum{\csname myrefnum@#1\endcsname}%
    \else
        \stepcounter{sideref}%
        \expandafter\xdef\csname myrefnum@#1\endcsname{\thesideref}%
        \edef\currentrefnum{\thesideref}%
    \fi
    % Registo de página robusto
    \zref@labelbyprops{curr:#1.\thesideref}{abspage}%
    \edef\this@page{\zref@extractdefault{curr:#1.\thesideref}{abspage}{\the\value{abspage}}}%
    % Link no texto
    \textsuperscript{\hyperref[sidemargin:#1.\this@page]{\textcolor{blue}{[\currentrefnum]}}}%
    % Conteúdo na margem (uma vez por página)
    \ifcsname pageref@#1@\this@page\endcsname
    \else
        \marginpar{%
            \raggedright
            \scriptsize
            \textcolor{blue}{Ref.~\currentrefnum}\,%
            \textcolor{darkgrey}{%
                \begin{NoHyper}
                    \citeauthor{#1}
                    (\citeyear{#1})
                    \textit{\usebibentry{#1}{title}} 
                \end{NoHyper}
            }%
            \label{sidemargin:#1.\this@page}%
        }%
        \global\expandafter\gdef\csname pageref@#1@\this@page\endcsname{true}%
    \fi
}

% 3. --- Side Figures ---
\newcommand{\sidefigure}[3][]{%
  \marginpar{%
    \centering
    \includegraphics[width=\marginparwidth]{#2}%
    \vspace{2pt} 
    % Aplicar cinzento à legenda da figura na margem
    \captionsetup{labelfont={color=blue,scriptsize}, font={color=darkgrey,scriptsize}}%
    \captionof{figure}{#3}%
    \ifx\\#1\\ \else \label{#1} \fi
  }%
}

% 4. --- Custom Sidetable Command (Com Label) ---
\newcommand{\sidetable}[5]{% #1: Label, #2: Cabeçalho 1, #3: Cabeçalho 2, #4: Cabeçalho 3, #5: Conteúdo
  \marginpar{%
    \raggedright
    \renewcommand{\arraystretch}{1.3}
    \captionsetup{
        type=table, 
        labelfont={color=blue,tiny}, 
        font={color=darkgrey,tiny}, 
        skip=4pt,
        justification=raggedright,
        singlelinecheck=false
    }
    \tiny
    \setlength{\tabcolsep}{2pt}
    \begin{tabularx}{\marginparwidth}{@{} l X l @{}}
      \hline
      \textbf{#2} & \textbf{#3} & \textbf{#4} \\ \hline
      #5 \\ \hline
    \end{tabularx}
    \captionof{table}{Nomenclatura.}
    \label{#1} % A label deve vir logo após o captionof
  }%
}


\makeatother

% Helper for figure referencing
\newcommand{\reffig}[1]{Fig.~\ref{#1}}

% --- Caption Styling ---
\captionsetup[figure]{
  labelformat=simple,
  labelsep=colon,
  name=Fig.,
  labelfont={color=blue,scriptsize},
  textfont={scriptsize},
  justification=raggedright,
  singlelinecheck=false
}

% =============================================================================
% CODE LISTINGS CONFIGURATION
% =============================================================================

\usepackage{listings}
\definecolor{codegreen}{rgb}{0,0.6,0}
\definecolor{codegray}{rgb}{0.5,0.5,0.5}
\definecolor{codepurple}{rgb}{0.58,0,0.82}
\definecolor{backcolour}{rgb}{0.95,0.95,0.92}

\lstset{
    backgroundcolor=\color{backcolour},   
    commentstyle=\color{codegreen},
    keywordstyle=\color{magenta},
    numberstyle=\tiny\color{codegray},
    stringstyle=\color{codepurple},
    basicstyle=\ttfamily\small, 
    breakatwhitespace=false,         
    breaklines=true,                 
    captionpos=b,                    
    keepspaces=true,                 
    numbers=left,                    
    numbersep=5pt,                  
    showspaces=false,                
    showstringspaces=false,
    showtabs=false,                  
    tabsize=2,
    frame=single,
    % Support for Portuguese/Spanish special characters in code
    literate={á}{{\'a}}1 {ã}{{\~a}}1 {â}{{\^a}}1 {à}{{\`a}}1 {Á}{{\'A}}1 {Ã}{{\~A}}1 {Â}{{\^A}}1 {À}{{\`A}}1
             {é}{{\'e}}1 {ê}{{\^e}}1 {É}{{\'E}}1 {Ê}{{\^E}}1
             {í}{{\'i}}1 {Í}{{\'I}}1
             {ó}{{\'o}}1 {õ}{{\~o}}1 {ô}{{\^o}}1 {Ó}{{\'O}}1 {Õ}{{\~O}}1 {Ô}{{\^O}}1
             {ú}{{\'u}}1 {Ú}{{\'U}}1
             {ç}{{\c{c}}}1 {Ç}{{\c{C}}}1
}