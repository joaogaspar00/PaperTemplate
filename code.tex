\section{Exemplos de Código}

Nesta seção, apresentamos como incluir fragmentos de código de diferentes linguagens de programação utilizando o pacote \texttt{listings}.

\subsection{MATLAB}
O código MATLAB é formatado com destaque para funções matemáticas e comentários com o símbolo \texttt{\%}.

\begin{lstlisting}[language=Matlab, caption={Cálculo de somatório em MATLAB}]
% Definir o limite
n = 100;
% Calcular a soma usando vetorização
soma = sum(1:n);
disp(['A soma total é: ', num2color(soma)]);
\end{lstlisting}

\subsection{Linguagem C}
Para C, o realce foca em diretivas de pré-processador e tipos de dados.

\begin{lstlisting}[language=C, caption={Estrutura básica em C}]
#include <stdio.h>

int main() {
    int n = 10;
    for(int i = 0; i < n; i++) {
        printf("Índice: %d\n", i);
    }
    return 0;
}
\end{lstlisting}

\subsection{Python}
Python utiliza uma sintaxe limpa, onde a indentação é fundamental. O \texttt{listings} preserva os espaços corretamente.

\begin{lstlisting}[language=Python, caption={Função recursiva em Python}]
def fibonacci(n):
    if n <= 1:
        return n
    else:
        return fibonacci(n-1) + fibonacci(n-2)

print(fibonacci(5))
\end{lstlisting}

\subsection{Pseudocódigo}
Para pseudocódigo, removemos as definições de linguagem para tratar o texto de forma genérica, mantendo a numeração de linhas.

\begin{lstlisting}[language={}, caption={Algoritmo de Troca}]
INÍCIO
    VARIAVEL temp, a, b
    LEIA a, b
    temp <- a
    a <- b
    b <- temp
    ESCREVA a, b
FIM
\end{lstlisting}

\sidenote{Pode ajustar as cores do código no preâmbulo alterando os valores de \texttt{codegreen} ou \texttt{codepurple}.}